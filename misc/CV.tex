\documentclass[12pt]{article}
\usepackage[utf8]{inputenc}
\usepackage{hyperref}
\usepackage{graphicx}
\usepackage{wrapfig}
\usepackage{geometry}
\usepackage{tikz}

\newcommand{\roundpic}[4][]{
  \tikz\node [circle, minimum width = #2,
    path picture = {
      \node [#1] at (path picture bounding box.center) {
        \includegraphics[width=#3]{#4}};
    }] {};}

\geometry{a4paper, top=2cm, bottom=2cm, left=1cm, right=1cm,  heightrounded, bindingoffset=5mm}

\hypersetup{
    colorlinks,
    citecolor=black,
    filecolor=black,
    linkcolor=black,
    urlcolor=black
}

\begin{document}

\section*{Riccardo Fusiello}

\subsection*{Chi sono}
Ciao, sono Riccardo Fusiello, ex studente dell'ITIS "Sen. Onofrio Jannuzzi" di Andria. Ho completato il percorso quinquennale nel 2021 con ottimi voti, in particolare nelle materie di indirizzo.\\
Ho conseguito la laurea triennale in Informatica presso l'Università degli Studi di Trento e attualmente frequento il corso di laurea magistrale in Artificial Intelligence presso l'Università degli Studi di Bari. La mia aspirazione è lavorare in un'azienda leader nel settore informatico o in una realtà che investa fortemente in innovazione tecnologica.\\
Attualmente lavoro alla System Project Srl, realtà locale che concilia questa mia aspirazione.\\\\
\href{https://riccardofus.github.io/Cavi-Website/}{\textbf{Sito Personale: https://riccardofus.github.io/Cavi-Website/}}\\
\textbf{Email}: \href{mailto:riccardo.fusiello02@gmail.com}{riccardo.fusiello02@gmail.com}\\
\textbf{Telefono}: (+39) 3895398639\\
\textbf{Data di nascita}: 25/11/2002\\
\textbf{Indirizzo}: Via Pericle 120, 76123 Andria (BT)

\subsection*{Esperienze Lavorative}
\textbf{06/09/2024 $-$ \dots} Sviluppatore Software presso System Project S.r.L, Andria
  \begin{itemize}
    \item Sviluppatore Odoo e Analista Funzionale (CRM, Vendite, Acquisti, Contabilità, \dots)
    \item Soluzioni di Intelligenza Artificiale integrate con Odoo
    \item Assistenza software e consulenza
    \item Vedi altro sul nostro \href{https://www.system-project.it/}{sito aziendale}: \textbf{https://www.system-project.it/}
  \end{itemize}
\textbf{2023 $-$ \dots} Lezioni private di materie scientifiche per studenti delle scuole superiori\\
\textbf{10/05/2024 $-$ 31/08/2024} Automazione di test psicologici con Google Suite\\
\textbf{05/02/2024 $-$ 29/03/2024} Tirocinio sul Machine Learning presso l'Università di Trento\\
\textbf{13/03/2023 $-$ 29/03/2024} IT Service Desk presso l'Università di Trento\\
\textbf{01/03/2022 $-$ 15/03/2022} Full Stack Developer presso Reply S.p.A (Remoto)\\
\textbf{15/06/2020 $-$ 15/09/2020} Full Stack Developer presso Nuvité S.r.l, Andria \\
\textbf{22/05/2019 $-$ 07/06/2019} Alternanza Scuola-Lavoro presso Exprivia S.p.A, Molfetta

\subsection*{Istruzione e Formazione}
\textbf{30/09/2024 $-$ \dots} Laurea Magistrale in Artificial Intelligence presso l'Università di Bari\\
\textbf{17/09/2021 $-$ 16/07/2024} Laurea Triennale in Informatica presso l'Università di Trento (106/110)\\
\textbf{15/09/2016 $-$ 15/06/2021} Diploma in Informatica e Telecomunicazioni presso ITIS "Sen. Onofrio Jannuzzi", Andria (100L/100)\\
\textbf{15/09/2019 $-$ 21/06/2020} Certificazione Cisco CCNA Routing and Switching: Introduction to Networks\\
\textbf{15/01/2021 $-$ 04/06/2021} Certificazione su Uso e Programmazione Robotica Comau Academy \\
\textbf{11/11/2024 $-$ 15/11/2024} Masterclass su Odoo presso Apulia Software

\subsection*{Lingue}
\textbf{Italiano}: Madrelingua\\
\textbf{Inglese}: Livello B1/B2, parlato quotidianamente e utilizzato nella scrittura tecnica

\subsection*{Linguaggi di Programmazione conosciuti} 
C/C++, Java, Python, JavaScript, CSS, HTML, Arduino, SQL, Shell, Kotlin, Jetpack Compose, Flutter, Angular

\subsection*{Progetti Personali}
\textbf{Shelfy}: la tua libreria digitale a portata di mano \\\\
\textbf{Esperimenti personali}: esperimenti con miriadi di pacchetti python per:
\begin{itemize}
  \item Automatizzare task personali 
  \item Realizzare modelli di intelligenza artificiali per task di vario tipo
  \begin{itemize}
    \item Generazione contenuto 
    \item Sistemi di raccomandazione conversazionali
    \item Computer Vision
  \end{itemize}
\end{itemize}

\vfill
\subsection*{}
Autorizzo il trattamento dei miei dati personali ai sensi del D. lgs. 196 del 30 giugno 2003 e del GDPR (Regolamento UE 2016/679).

\end{document}